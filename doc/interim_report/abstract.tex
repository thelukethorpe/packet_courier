\thispagestyle{empty}
\hrule\hrule\hrule\hrule %When put line outside table and we use \hline in a table
\begin{center}
    \textbf{\large Abstract}
    \\
\end{center}

\noindent
\lettrine{M}{odern} day infrastructure is almost entirely built upon concepts of long-range communication. If any of
us wish to contact a family member, friend or colleague on the other side of the globe, then we scarcely hesitate to
use one of the many services that enable us to do so. In fact, it is nuisance if we encounter technical issues along
the way. This is somewhat of a social paradox, however, because the various algorithms and interweaving components
that underpin such powerful technologies are enormously complex. These fundamental protocols are expected to reliably
deliver a lengthy contract of communicative obligations, whilst deftly navigating the various pitfalls and obstacles
that are characteristic of the internet. Indeed, such implementations are difficult to test; finding a virtual
landscape that closely imitates the true chaos of the web is a hard problem. Of course, one could simply use the
internet itself, setting up machines based in London, Paris and Amsterdam, say. This is a costly operation, however,
and one that is inflexible, with limited scope for a satisfying variety of test cases. Thus, there is a clear need for
simulation-based testing.

This project endeavours to build a high-calibre tool that can be used to run distributed computing solutions in a
highly configurable simulated environment. In turn, the end product will enable developers, network engineers and
students alike to fine tune a network exactly to their liking and let their algorithms loose in a way that is
self-contained, realistic and resource inexpensive. The ultimate goal is to provide those with a curisoity for
distributed computer systems with the means to tinker around with the technologies as though they were running in the
real world.


\vfill
\hrule\hrule\hrule\hrule
\clearpage

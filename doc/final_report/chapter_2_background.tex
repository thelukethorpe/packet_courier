\section{Existing Tools}

\subsection{tc-netem}

The network emulation tool known as \texttt{tc-netem} is a staple of the Linux operating system\cite{tc_netem_wiki,
    tc_netem_8_man,tc_netem_src}. It allows users to attach a virtual packet pipeline to a local network device and
offers the following features:
\begin{itemize}
    \item \textbf{Bandwidth throttling} \\
    Constrains bitrate to a specified upper limit.
    \item \textbf{Packet limiting} \\
    Enforces that only a certain number of packets can be enqueued at any one time.
    \item \textbf{Packet delay} \\
    Imbues packets with an artificial latency in keeping with a chosen distribution (uniform, normal, Pareto or
    Pareto-normal).
    \item \textbf{Packet loss} \\
    Drops packets according to a parameterized strategy (a Bernoulli distribution, a 4-state Markov chain or the
    Gilbert-Elliott model\cite{ge_model}).
    \item \textbf{Packet corruption} \\
    Samples a Bernoulli distribution and flips a random bit on success.
    \item \textbf{Packet duplication} \\
    Samples a Bernoulli distribution and duplicates the given packet on success (notably can be generalised to a
    Poisson distribution rather trivially).
    \item \textbf{Packet slotting} \\
    Partitions packets into buckets referred to as ``slots''. A slot will phase in and out of being active in its
    packet transmission, which can be used to emulate burst behaviours.
\end{itemize}

In this way, the sophistication and variety of \texttt{tc-netem}'s feature-set is clear for all to see. Indeed, the
tool appears to be so rich in its capabilities that one could build a packet pipeline to reflect any arbitrary
network conditions, and as such, \texttt{tc-netem} has set the bar when it comes to the depth of possible
configuration. It is not without its downsides, though, the first of which being that it is a Linux native
application. There are analogous products for Mac in the form of \texttt{ns-2}\cite{ns_2_man, ns_2_wiki} and
\texttt{Network Link Conditioner}\cite{nlc}, and similarly \texttt{ipfw}\cite{ipfw,ipfw_man} for Windows, which are
deserving of investigation on their own merits. Yet, unless an OS adaptive approach is undertaken at the design
stage, their existence alone does little to console their lack of platform agnosticism.

\texttt{tc-netem} also happens to be deficient when it comes to the ``quality of life'' of a prospective user. If basic
packet modulation is all that is required, then \texttt{tc-netem} does the job excellently. Consider a game developer
who wishes to ensure that their game runs relatively smoothly in spite of latency and jitter, but does not necessarily
have the material means to establish a genuine WAN\cite{wan_cisco} connection between two game instances. In this
case, \texttt{tc-netem} can be utilised to full effect; with a simple, one-time terminal command, a packet pipeline
can be established between two local game clients.

Now consider a complex and innovative game being developed by a
studio with over one thousand employees. The scope of the game is so vast that it requires computation to be split over
multiple machines forming an intricate topology, which in turn will serve hundreds of client requests every second.
Although \texttt{tc-netem} could in principle deliver the testbed demands of such a system, it almost certainly
could not be done by hand, so to speak. Instead, it would necessitate some kind of managment software that would wrap
the \texttt{tc-netem} functionality and present programmers with a palatable interface, only then to \emph{still}
fall foul of platform dependency concerns; a non-trivial set of concerns for an innately multi-platform exercise such
has video game development.

It should be made explicitly clear that this scenario is not illusory (in case it did happen to portray itself as a
straw-man giddily clapping at the mere proposition of the research to come). There is much activity within this
space, as demonstrated by companies such as Microsoft, Heroic Labs, Improbable developing tools and services that
enable the deployment of highly distributive and pervasive video games\cite{microsoft_playfab, heroic_labs_nakama,
    improbable_spatialos, improbable_spatialos_unreal_gdk_github}. As of Janurary 2022, Improbable's child company,
Midwinter Entertainment, is on the verge of releasing \emph{Scavengers}\cite{improbable_spatialos_scavengers}, a
battle-royal style shooter built using the \texttt{SpatialOS} framework\cite{improbable_spatialos,
    improbable_spatialos_unreal_gdk_github}.


\section{Related Technologies}

TODO

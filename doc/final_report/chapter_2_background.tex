\section{Existing Tools}

\subsection{tc-netem}

The network emulation tool known as \texttt{tc-netem} is a staple of the Linux operating system\cite{tc_netem_wiki,
    tc_netem_8_man,tc_netem_src}. It allows users to attach a virtual packet pipeline to a local network device and
offers the following features:
\begin{itemize}
    \item \textbf{Bandwidth throttling} \\
    Constrains bitrate to a specified upper limit.
    \item \textbf{Packet limiting} \\
    Enforces that only a certain number of packets can be enqueued at any one time.
    \item \textbf{Packet delay} \\
    Imbues packets with an artificial latency in keeping with a chosen distribution (uniform, normal, Pareto or
    Pareto-normal).
    \item \textbf{Packet loss} \\
    Drops packets according to a parameterized strategy (a Bernoulli distribution, a 4-state Markov chain or the
    Gilbert-Elliott model\cite{ge_model}).
    \item \textbf{Packet corruption} \\
    Samples a Bernoulli distribution and flips a bit at random on success.
    \item \textbf{Packet duplication} \\
    Samples a Bernoulli distribution and duplicates the given packet on success (notably can be generalised to a
    Poisson distribution rather trivially).
    \item \textbf{Packet slotting} \\
    Partitions packets into buckets referred to as ``slots''. A slot will phase in and out of being active in its
    packet transmission, which can be used to emulate burst behaviours.
\end{itemize}

In this way, the sophistication and variety of \texttt{tc-netem}'s feature-set is clear for all to see. Indeed, the
tool appears to be so rich in its capabilities that one could build a packet pipeline to reflect any arbitrary
network conditions, and as such, \texttt{tc-netem} has set the bar when it comes to the depth of possible
configuration. It is not without its downsides, though, the first of which being that it is a Linux native
application. There are analogous products for Mac in the form of \texttt{ns-2}\cite{ns_2_man, ns_2_wiki} and
\texttt{Network Link Conditioner}\cite{nlc}, and similarly \texttt{ipfw}\cite{ipfw,ipfw_man} for Windows.


\section{Related Technologies}

TODO

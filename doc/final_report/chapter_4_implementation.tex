\section{Project Management}

\subsection{Version Control}

Packet Courier was developed using Git\cite{git} as a version control system and the code repository is hosted on
GitHub\cite{github, packet_courier} at \url{https://github.com/thelukethorpe/packet_courier}. There is no special
reason for either of these choices; they are simply industry standards being used for a project that doesn't
innately demand bespoke tooling (unlike data-heavy ventures such as game development, which might benefit from a tool
like Perforce\cite{perforce, perforce_vs_git}).

\subsection{Ticket Tracking}

GitHub Issues\cite{github_issues} powers Packet Courier's ticket tracking. Tickets, or \emph{issues}, as per the
GitHub nomenclature, are categorised based on one or more of the following custom tags:
\begin{itemize}
    \item \textbf{Bug:} \emph{Something isn't working.}
    \item \textbf{CI:} \emph{Change to pipeline.}
    \item \textbf{Documentation:} \emph{Improvements or additions to documentation.}
    \item \textbf{Feature:} \emph{New feature or request.}
    \item \textbf{Optimization:} \emph{Improvements in performance.}
    \item \textbf{Refactor:} \emph{Tech-debt, structural change of quality-of-life improvement.}
    \item \textbf{Testing:} \emph{Improvements or additions to test suite.}
    \item \textbf{Won't Fix:} \emph{This will not be worked on.}
\end{itemize}

Issues are further organised using a Kanban board\cite{kanban_board}. As the \emph{documentation} tag suggests,
this board is not only used for development purposes, in fact, it has been used to help manage the project
holistically, including the writing of this very report, as shown in
Figure~\ref{fig:chapter_4_implementation-github_kanban_board}.

\begin{figure}[!h]
    \includegraphics[width=\textwidth]{images/chapter_4_implementation/github_kanban_board}
    \centering~\caption{Packet Courier's Project Kanban Board\cite{packet_courier}.}
    \label{fig:chapter_4_implementation-github_kanban_board}
\end{figure}

Pull-requests are opened for every merge to the \texttt{main} branch and linked to the associated ticket number.
Branches are named with a uniform and consistent structure, prefixed by the issue number followed by a brief subtitle
of the ticket delimited by hyphens. Despite being an individual project, this repository has been maintained to
industry standards of software engineering practices for organisational reasons, and has in turn benefited massively
from it.

\begin{figure}[!h]
    \includegraphics[width=\textwidth]{images/chapter_4_implementation/github_pull_request}
    \centering~\caption{An example of a pull-request in the Packet Courier repository\cite{packet_courier}.}
    \label{fig:chapter_4_implementation-github_pull_request}
\end{figure}

\newpage

\subsection{Programming Languages}

Java 8\cite{java_8} was chosen as the foundational language for Packet Courier due to:
\begin{itemize}
    \item Its excellent breadth of supported platforms\cite{java_8_support}, which lends itself well to objective 4.a).
    \item The vast number of popular and easily importable libraries available to it\cite{java_relevance}.
    \item Its object-oriented nature, which maps nicely onto the abstract semantics laid out in the design section.
    \item Its relatively high levels of time-efficiency, as shown in
    Figure~\ref{fig:chapter_4_implementation-programming_language_comparison}.
    \item Its suite of concurrency abstractions\cite{java_util_concurrent}.
\end{itemize}

\begin{figure}[!h]
    \includegraphics[width=\textwidth]{images/chapter_4_implementation/programming_language_comparison}
    \centering~\caption{A comparison of programming languages based on energy usage, time and memory
    efficiency\cite{programming_language_efficiency}.}
    \label{fig:chapter_4_implementation-programming_language_comparison}
\end{figure}

Although languages such as C and C++ outperform Java across all metrics displayed by
Figure~\ref{fig:chapter_4_implementation-programming_language_comparison}, they are needlessly low-level to the point
where it would likely become a hindrance. Even a cursory investigation into the nature of socket programming in C++
revealed a perfect case study as to the benefits of Java for a problem statement like Packet Courier's. It turns out
that Windows does not natively support the standard C++ socket specification, \texttt{sys/socket.h}, meaning that
\texttt{winsock2.h} must be used instead\cite{socket_vs_winsock}. This would then mean either going to the effort of
implementing two different platform-dependent solutions, or using a library that promises platform agnosticism. There
are many such libraries however\cite{c++_socket_libraries}, placing a substantial research burden on the development
of what should be a fairly simple component. C++ furthermore has been shown to have several non-trivial cross-platform
and cross-compiler discrepancies for even the most basic code\cite{c++_cross_compiler_differences,
    c++_statistical_differences, c++_random_differences, c++_filesystem_differences}.

A key benefit of Java is that it smooths out cross-platform issues, providing developers with a clean, uniform and
highly reliable set of semantics that transcend the nuances of the native operating system and hardware
micro-architecture. Indeed, Java handily has a \texttt{DatagramSocket} class\cite{java_DatagramSocket} which has
proven extremely useful in implementing Packet Courier's standalone emulator. In this way, external Java libraries
are only called-in for more heavyweight or highly specific work, as opposed to routine or menial implementations.

Java also strikes a Goldilocks-esque balance\cite{goldilocks_effect} between low-level languages like C++ and
languages that are arguably even more abstract than Java, such as Python. Whilst Packet Courier does leverage Python
3\cite{python_3} for basic client-server scripts to demonstrate the functionality of the emulator, production code
consists exclusively of Java. Python's dynamic typing\cite{python_typing}, poor
performance\cite{programming_language_efficiency} and inefficient multi-threading and synchronisation
primitives\cite{python_gil} are the main reasons why it wasn't used more prominently within Packet Courier.

\subsection{Build and Dependency Management}

Packet Courier uses Apache Maven 3.6.3\cite{maven} to manage its build phases and dependencies. Maven is convienent,
lightweight, widely supported, boasts a huge repository of libraries\cite{maven_repository} and with a single
command can compile the project into a portable \texttt{.jar} file that can be used as an executable binary or a Java
library.

\subsubsection{JUnit 4}

JUnit 4\cite{juint4} is an industry standard for conducting tests in Java, with an estimated 30.7\% of Java projects on
GitHub using JUnit (according to a study done in 2013)\cite{java_library_popularity}. Naturally, the Packet Courier
unit tests are implemented using JUint 4.13 and are run on each Maven build.

\subsubsection{AssertJ}

AssertJ\cite{assert_j} is used in conjunction with JUnit to provide unit tests with expressive, human-readable checks
such as:

\begin{lstlisting}[language=Java]
public class PacketTest {
  // JUnit 4 test.
  @Test
  public void testPacketStringConversion() {
    String message = "hello there!";
    Packet messageAsPacket = Packet.of(message);
    // AssertJ assertion reads like an English sentence.
    assertThat(messageAsPacket.tryParse()).hasValue(message);
  }

  // Some more tests...
}
\end{lstlisting}

The following paragraph from a \emph{JavaZone} article summarises why AssertJ was favoured over the JUnit default
assertion framework known as Hamcrest\cite{assert_j_vs_hamcrest}:
\begin{quote}
    \emph{``AssertJ is not as well-known as Hamcrest, but at the same time, its popularity has been growing pretty
    fast over the last few years. As opposed to Hamcrest’s classic assertion syntax, which was inherited from the
    default Java testing framework JUnit, the main idea of AssertJ is that it provides fluent syntax. The main goal
    of that is to improve code readability. It’s worth mentioning that AssertJ is a fork of the FEST Assert project,
        which was the first step of AssertJ creation.''}
\end{quote}

\subsubsection{Google Protocol Buffer}

Google Protocol Buffer\cite{google_protobuf} is a multi-faceted library that is supported across multiple languages,
but is used by Packet Courier to parse configuration files into Java objects. There are many file parsing libraries
for specific file formats, such as Google Gson, a Java JSON parsing library\cite{google_gson}. The unique selling
point of Google Protocol Buffer is its neatly compartmentalised workflow:
\begin{enumerate}
    \item Write a \texttt{.proto} file that defines the expected file structure. See
    Code-Snippet~\ref{code:google_protocol_buffer_example} for an example.
    \item Compile the \texttt{.proto} file using Maven to generate the corresponding parse-tree in Java.
    \item Choose a parser such as \texttt{JsonFormat} to generate a parse-tree from an input file.
    \item Perform a semantic pass over the parse-tree, i.e.: completing basic sanity checks whilst converting the
    raw syntactic form of the parse-tree into something more abstract that can be assimilated into the core APIs.
\end{enumerate}

\begin{lstlisting}[language=protobuf2,style=protobuf,caption={An example of a \texttt{.proto} file that encodes for
an \texttt{AddressBook}. A data file that followed this syntactic structure could then be read into memory as an
\texttt{AddressBookProtos} Java class, ready for further abstraction.},
    label={code:google_protocol_buffer_example},captionpos=b]
    syntax = "proto2";

    package tutorial;

    option java_multiple_files = true;
    option java_package = "com.example.tutorial.protos";
    option java_outer_classname = "AddressBookProtos";

    message Person {
        optional string name = 1;
        optional int32 id = 2;
        optional string email = 3;

        enum PhoneType {
            MOBILE = 0;
            HOME = 1;
            WORK = 2;
        }

        message PhoneNumber {
            optional string number = 1;
            optional PhoneType type = 2 [default = HOME];
        }

        repeated PhoneNumber phones = 4;
    }

    message AddressBook {
        repeated Person people = 1;
    }
\end{lstlisting}

\subsubsection{Protocol Buffers Protobuf Maven Plugin}

An unusual quirk of the Google Protocol Buffer Java library is that it doesn't compile all of the requisite Java
classes into the final \texttt{.jar}; instead it assumes that they will be loaded into the project as part of a
separate library. The \texttt{protoc-jar-maven-plugin}\cite{protoc_jar_maven_plugin} helps to work around this issue
by adding an extra compilation step at compile time to plug this gap.

\subsection{Continuous Integration}

The Packet Courier repository uses GitHub actions\cite{github_actions} to implement a continuous integration
pipeline\cite{aws_ci} where code is linted and tested, both of which are done using Maven. Linting is enforced using
\texttt{fmt-maven-plugin}\cite{fmt_maven_plugin}, which provides users with an automated way to check for and fix Java
code that does not adhere to the Google Java Style Guide\cite{google_java_format, google_java_style_guide}. Testing
is simply done via the Maven command:
\begin{quote}
    \texttt{mvn '-Dtest=**.*Test' test --no-transfer-progress}.
\end{quote}


\newpage


\section{API Overview}

\subsection{Repository Structure}

\begin{figure}[!h]
    \includegraphics[width=\textwidth]{images/chapter_4_implementation/repository_structure}
    \centering~\caption{A diagram showing the directory tree of the repository's source code. The tests reside in
    \texttt{src/test} and are analogously structured.}
    \label{fig:chapter_4_implementation-repository_structure}
\end{figure}

\subsection{Statistical Distribution API}

TODO

\subsection{Network Condition API}

TODO

\subsection{Worker API}

TODO

\subsection{Node API}

TODO

\subsection{Mail API}

TODO

\subsection{Simulation API}

TODO

\subsection{Miscilaneous}

TODO


\section{Standalone Emulator}

\subsection{Key Components}

TODO

\subsection{Protobuf Integration}

TODO


\section{Debugging Features}

\subsection{Process Monitor}

TODO

\subsection{Crash Dumps}

TODO

\subsection{Meta Logging}

TODO

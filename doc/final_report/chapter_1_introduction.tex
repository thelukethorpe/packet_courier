\section{Motivation}

Distributed computing is naturally a field that explores non-trivial levels of parallelism on a large scale, and as
such, any fledgling distributed system demands a testbed that epitomizes these properties. It then follows that
network simulation is likely to be the apparatus of choice, notwithstanding the possibility of expending serious
resource to conduct ``live'' tests over an existing physical topology, perhaps even the internet itself.

In spite of this, designing and building a simulation tool is an inherently multivariate and sophisticated undertaking
in its own right. A packet can experience a great deal during its lifetime. Indeed, most packets are ultimately
buffered, serialized into bits, sent over a wire and are then thrust into a hellish domain whereby they can be
subject to the horrors of corruption, duplication, reordering, delay and possibly even outright elimination. To boot,
they can sometimes go airborne during this process, having been shot out of a wireless router, only to land right
back into the terror-laden realm that we so innocently call ``the internet''.

It is in this way that good simulation software should be capable of accurately mimicking the these behaviours
according to any arbitrary configuration. In fact, the potential for accute granularity that a simulator offers makes
it far superior to any ``real'' test scenario, insofar as it can be moulded to investigate specific nuances within a
particular algorithm or protocol. For example, most low-level distributed technologies must find ways to overcome
both packet loss and corruption. Although these are independant issues, they tend to co-exist in practice, and are
often correlated phenomena, which makes isolated testing difficult. This problem is absent in a simulated context
however, since virtual environments can be tuned to boast whatever characteristics are deemed necessary for
the given exercise. In resolution of the aforementioned example, two simulations can be run at the descretion of the
developers, one where corruption is the only danger present, along with a similar configuration that drops packets
instead.


\section{Objectives}

TODO


\section{Contributions}

TODO

\thispagestyle{empty}
\hrule\hrule\hrule\hrule %When put line outside table and we use \hline in a table
\begin{center}
    \textbf{\large Abstract}
    \\
\end{center}

\lettrine{D}{istributed} computing is naturally a field that explores non-trivial levels of parallelism on a large
scale, and as such, any fledgling distributed system demands a testbed that epitomizes these properties. Network
simulation is likely to be the apparatus of choice when building such a testbed, yet this thesis finds that even
some of the most popular frameworks are slightly lacking when it comes to ease of use and platform portability.

The proposed solution produced by this thesis comes in the form of \emph{Packet Courier}, a Java library that doubles
up as both a Java simulation framework and a standalone emulator. In this way, a prospective user can either
configure a distributed network simulation programmatically using Java, or they can run the compiled Jar like an
executable in conjunction with a JSON-style configuration file to emulate a distributed network. The benefit of
Packet Courier's emulation component is that it allows users to run arbitrary network code: their distributed
algorithm can leverage any processes that can be run from a command prompt, whether that be a Python script, a
C++ program or a high-level tool such as Kubernetes. Packet Courier achieves this by manipulating packets during
transit by creating an intermediate routing layer between nodes in the network. For example, if Alice sent a packet
to Bob, then the Packet Courier framework would silently intercept this packet and process it as per the given
configuration. Users can define network conditions that will govern how packets behave when travelling from node to
node, including properties such as packet latency, loss and corruption.


\vfill
\hrule\hrule\hrule\hrule
\clearpage

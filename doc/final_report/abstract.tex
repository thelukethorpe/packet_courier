\thispagestyle{empty}
\hrule\hrule\hrule\hrule %When put line outside table and we use \hline in a table
\begin{center}
    \textbf{\large Abstract}
    \\
\end{center}

\noindent
The proposed solution produced by this thesis comes in the form of \emph{Packet Courier}, a Java library that doubles
up as both a Java simulation framework and a standalone emulator. In this way, a prospective user can either
configure a distributed network simulation programmatically using Java, or they can run the compiled Jar like an
executable in conjunction with a JSON style configuration file to emulate a distributed network. The benefit of
Packet Courier's emulation component is that it allows users to run arbitrary network code: their distributed
algorithm can leverage any processes that can be run from the command line, whether that be a Python script, a
C++ program or a high-level tool such as Kubernetes.

Packet Courier is able to manipulate packets during transit by creating an intermediate routing layer between nodes
in the network. For example, if Alice sent a packet to Bob, then the Packet Courier framework would silently
intercept this packet and process it as per the given configuration. Users can define network conditions that will
govern how packets behave when travelling from node to node, including properties such as packet latency, loss and
corruption.


\vfill
\hrule\hrule\hrule\hrule
\clearpage

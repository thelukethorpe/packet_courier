\thispagestyle{empty}
\hrule\hrule\hrule\hrule %When put line outside table and we use \hline in a table
\begin{center}
    \textbf{\large Abstract}
    \\
\end{center}

\noindent
\lettrine{M}{odern} day infrastructure is almost entirely built upon concepts of long-range communication. If any of
us wish to contact a family member, friend or colleague on the other side of the globe, then we scarcely hesitate to
use one of the many services that enable us to do so. In fact, it is nuisance if we encounter technical issues along
the way. Yet, this is somewhat of a social paradox, since the various algorithms and interweaving components
that underpin such powerful technologies are enormously complex. These fundamental protocols are expected to reliably
deliver a lengthy contract of communicative obligations, whilst deftly navigating the various pitfalls and obstacles
that are characteristic of the internet. Indeed, such implementations are difficult to test; finding a virtual
landscape that closely imitates the true chaos of the web is a hard problem. Of course, one could simply use the
internet itself, setting up machines based in London, Paris and Amsterdam, say. This is a costly operation, however,
and one that is inflexible, with limited scope for a satisfying variety of test cases. Thus, there is a clear need for
simulation-based testing.

The proposed solution produced by this thesis comes in the form of \emph{Packet Courier}, a Java library that doubles
up as both a Java simulation framework and a standalone emulator. In this way, a prospective user can either
configure a distributed network simulation programmatically using Java, or they can run the compiled Jar like an
executable in conjunction with a JSON style configuration file to emulate a distributed network. The benefit of
Packet Courier's emulation component is that it allows users to run arbitrary network code: their distributed
algorithm can leverage any processes that can be run from the command line, whether that be a Python script, a
C++ program or a high-level tool such as Kubernetes.

Packet Courier is able to manipulate packets during transit by creating an intermediate routing layer between nodes
in the network. For example, if Alice sent a packet to Bob, then the Packet Courier framework would silently
intercept this packet and process it as per the given configuration. Users can define network conditions that will
govern how packets behave when travelling from node to node, including properties such as packet latency, loss and
corruption.


\vfill
\hrule\hrule\hrule\hrule
\clearpage

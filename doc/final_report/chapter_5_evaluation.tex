\section{Unit Testing}\label{section:unit_testing}

TODO


\section{System Testing}\label{section:system_testing}

TODO


\section{User Feedback}\label{section:user_feedback}

TODO


\section{Analysis of Objectives}\label{section:analysis_of_objectives}

\subsection{Network Topology Design Interface}\label{subsection:network_topology_design_interface}

\emph{Objective 1: An interface to design an arbitrary network topology.}

\subsection{Network Conditions Configuration Suite}\label{subsection:network_conditions_configuration_suite }

\emph{Objective 2: A configuration suite to define how packets are manipulated during message-passing.}

\subsection{Simulation of a Distributed Algorithm}\label{subsection:simulation_of_a_distributed_algorithm}

\emph{Objective 3: A mechanism to run a distributed algorithm across a virtual network topology which reflects the
properties described by the user as per objectives 1) and 2).}

\subsection{Platform Agnosticism}\label{subsection:platform_agnosticism}

\emph{Objective 4.a: Basic usage of the tool should not depend on the operating system or hardware being used.} \\
\emph{Objective 4.b: Users should not be pigeonholed into working with a particular programming language in order to
simulate their solution.}

\subsection{Plug-and-playability}\label{subsection:plug_and_playability}

\emph{Objective 5.a: Users should need minimal domain-specific knowledge in order to use the full set of features on
offer.} \\
\emph{Objective 5.b: The prospective tool should be able to mimic a real network in a way that minimises bespoke set-up,
    i.e.: if a user normally tests their distributed algorithm using real computers connected over a physical
    network, then transitioning to using the prospective tool should be more or less seamless.}

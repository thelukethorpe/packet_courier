In the early stages of Packet Courier's development, the mission statement was \emph{``to provide those with a
curiosity for distributed computer systems with the means to tinker around with network technologies as though they
were running in the real world''}. In fact, it was this very motto that inspired the more formal objectives defined
in Section~\ref{section:objectives}. Hopefully this report has done its due diligence to demonstrate beyond all doubt
that Packet Courier is a tool that truly embodies this sentiment. It provides developers, network engineers and
students alike to fine tune a network exactly to their liking and let their algorithms loose in a way that is
self-contained, realistic and resource inexpensive. Packet Courier has also proven to be a high-calibre piece of
software, insofar as it can emulate 100 nodes exchanging 5,000 packets per second in real time without so much as
adding 1ms of overhead latency to each packet (on the right hardware, of course). Many would consider a framework
capable of these feats to be worthy of usage in serious software and network engineering ventures.

Although Packet Courier was found to have demonstrably achieved most of its objectives in
Section~\ref{section:analysis_of_objectives}, there was ample scope for improvement. A mixture of prospective
features, performance improvements, documentation enhancements and bugfixes lie ahead in Packet Courier's future,
especially given that it will be released as an open source project on the 21\textsuperscript{st} of June 2022.


\section{Future Work}\label{section:future_work}

\subsection{Fix Bandwidth Throttling Algorithm}\label{subsection:fix_bandwidth_throttling_algorithm}

TODO

\subsection{Add Support for YAML Configuration Files}\label{subsection:support_for_yaml_configuration_files}

TODO

\subsection{Improve Configuration File Sanitization}\label{subsection:improve_configuration_file_sanitization}

TODO

\subsection{Add Platform-Specific Testing to CI}\label{subsection:add_platform_specific_testing_to_ci}

TODO

\subsection{Integrate System Testing into CI}\label{subsection:integrate_system_testing_into_ci}

TODO

\subsection{Implement GUI}\label{subsection:implement_gui}

TODO

\subsection{Add Integration Testing using Mockito}\label{subsection:add_integration_testing_using_mockito}

TODO

\subsection{Miscellaneous Features}\label{subsection:miscellaneous_features}

TODO

\subsection{Miscellaneous Bugfixes}\label{subsection:miscellaneous_bugfixes}

TODO

\lettrine{I}{n} the early stages of Packet Courier's development, the mission statement was \emph{``to provide those
with a curiosity for distributed computer systems with the means to tinker around with network technologies as though
they were running in the real world''}. In fact, it was this very motto that inspired the more formal objectives
defined in Section~\ref{section:objectives}. Hopefully this report has done its due diligence to demonstrate beyond
all doubt that Packet Courier is a tool that truly embodies this sentiment. It provides developers, network engineers
and students alike to fine tune a network exactly to their liking and let their algorithms loose in a way that is
self-contained, realistic and resource inexpensive. Packet Courier has also proven to be a high-calibre piece of
software, insofar as it can emulate 100 nodes exchanging 5,000 packets per second in real time without so much as
adding 1ms of overhead latency to each packet (on the right hardware, of course). Many would consider a framework
capable of these feats to be worthy of usage in serious software and network engineering ventures.

Although Packet Courier was found to have demonstrably achieved most of its objectives in
Section~\ref{section:analysis_of_objectives}, there was ample scope for improvement. A mixture of prospective
features, performance improvements, documentation enhancements and bugfixes lie ahead in Packet Courier's future,
especially given that it will be released as an open source project on the 21\textsuperscript{st} of June 2022.


\section{Future Work}\label{section:future_work}

\subsection{Fix Bandwidth Throttling Algorithm}\label{subsection:fix_bandwidth_throttling_algorithm}

\lettrine{A}{s} discussed in Section~\ref{subsubsection:throttle_analysis}, the bandwidth throttler seems to be the
only core component of Packet Courier's network condition semantics that is not performing as advertised in
Section~\ref{subsection:network_condition_semantics}. As part of the analysis conducted in
Section~\ref{subsubsection:throttle_analysis}, Packet Courier's throttling algorithm appears to face a bin-packing
issue where packets are imbued with a delay (and thus throttled) indiscriminately, even if traffic is relatively
light. As such, the current implementation needs to be upgraded to ``balance the scales the other way'' during
periods of low activity, so to speak. This could be done by constantly measuring the rate of packet flow and
adjusting the delay packets receive based on that, therefore resulting in low latency during periods of low traffic,
resulting in packets travelling through the pipeline quickly, i.e.: minimal throttling, and conversely, higher
latencies when the packet flow is approaching the maximum bandwidth, i.e.: maximum throttling.

\subsection{Add Support for YAML Configuration Files}\label{subsection:support_for_yaml_configuration_files}

\lettrine{E}{xactly} 50\% of survey participants made this precise suggestion as per the results detailed in
Section~\ref{subsection:survey_results}. Others who participated in user feedback were indirectly in favour of this
feature, claiming that the \texttt{.courierconfig} file format could be cleaner. Although the Google Protocol Buffer
API only seems to support JSON parsing\cite{com_google_protobuf_util}, there are plenty of ways to convert YAML into
JSON\cite{yaml_to_json}. As such, this would be a reasonably simple feature to add.

\subsection{Improve Configuration File Sanitization}\label{subsection:improve_configuration_file_sanitization}

\lettrine{A}{nother} common criticism seen in Section~\ref{section:user_feedback} is that the errors produced by the
\texttt{.courierconfig} parser are not always very helpful. One way to mitigate this is to add more custom semantic
checks into \texttt{PacketCourierSimulationConfigurationProtoParser<NodeInfo>} to ensure that more abstract errors
such as specifying a negative number of clients for a star topology isn't possible. Most of the problems encountered
by users came from the Google Protocol Buffer JSON parser itself. There are three possible ways to tackle this issue:
\begin{itemize}
    \item Write a custom parser. This would also make the Google Protocol Buffer dependency redundant. It would be
    quite a considerable amount of work, however.
    \item Many parsers integrate some kind of ``error listener'' or ``observer pattern''\cite{observer_pattern} which
    invokes a method when a parser error is encountered. This could help Packet Courier wrap Google Protocol Buffer
    errors and make them more context specific and palatable to the user.
    \item Make a full switch over to a YAML format and squeeze some kind of error checker into the YAML to JSON
    conversion process. This smells as though it would be fairly fragile, however.
\end{itemize}

\newpage

\subsection{Add Platform-Specific Testing to CI}\label{subsection:add_platform_specific_testing_to_ci}

\lettrine{O}{ne} of the areas were Packet Courier didn't quite live up to its expectations was in
Section~\ref{subsection:survey_results} when it was discovered that there were platform-specific issues on Windows
and MacOS. As Packet Courier evolves, it cannot rely on user trials to uncover these kinds of problems. The best way
to avoid them in the future would be to integrate platform testing into the CI. As per
Section~\ref{subsection:continuous_integration}, CI is currently done using GitHub actions which are
Dockerized\cite{dockerizing} and thus can be run on virtually any platform. As such, integration and system testing
should be done across all major platforms to ensure that the production branch works on Windows, MacOS and Linux. In
this way, ensuring platform agnosticism becomes part of the workflow rather than an afterthought.

\subsection{Integrate System Testing into CI}\label{subsection:integrate_system_testing_into_ci}

\lettrine{A} huge amount of tooling is currently sat in the Packet Courier repository without any form of automation
to take advantage of it. It would only seem sensible to have some sort of runner or virtual machine run a selection
of system tests once or twice a day to ensure that Packet Courier is always performing well end-to-end. Note that any
virtual machine would likely need to be more powerful than the one used for the system testing in
Section~\ref{section:system_testing}, because if the control data is anything to go by, the system tests would
constantly be failing due to hardware constraints.

\subsection{Implement a GUI}\label{subsection:implement_gui}

\lettrine{T}{he} suggestion of a graphical user interface popped up a few times in Section~\ref{section:user_feedback}.
This probably falls fairly low down on the list of priorities due to its more superficial nature, it is still a very
valid potential feature, and one that would add value for more casual users of the tool or perhaps even in education.
A framework such as JavaFX\cite{java_fx} would work nicely, or possibly even some kind of web-based UI.

\subsection{Add Integration Testing using Mockito}\label{subsection:add_integration_testing_using_mockito}

\lettrine{C}{urrently} Packet Courier only does unit and system testing. Integration testing would be useful to test
isolated parts of the APIs such as the packet filters and packet pipelines. One way this could be achieved is by using
Mockito\cite{mockito} to ``mock'' the components adjacent to pipelines and filters, whereby ``fake'' packets are used
to verify the correctness of behaviour.

\subsection{Miscellaneous Features}\label{subsection:miscellaneous_features}

\begin{itemize}
    \item Enable users to specify a ``custom'' virtual-clock interval, i.e.: 3.65ms.
    \item Implement a Pareto distribution.
    \item Implement a lat/long heuristic where nodes are given coordinates and their network conditions are estimated
    based on that, i.e.: if two locations are closer together based on the Haversine formula\cite{haversine_formula},
    then their latency will be lower.
    \item Allow nodes to connect to each other on-the-fly in real time.
    \item Allow new nodes to spawn in real time.
    \item Add more complex examples for users to play around with.
\end{itemize}

\subsection{Miscellaneous Bugfixes}\label{subsection:miscellaneous_bugfixes}

\begin{itemize}
    \item Packet Courier's \texttt{NormalDistribution} implementation takes advantage of Java's
    \texttt{Random}\cite{java_Random} class to generate a uniformly random \texttt{double} in the interval $[0, 1)$.
    However, the \texttt{sample} method in \texttt{NormalDistribution} will throw an exception if this number happens
    to be bang-on 0, which is 1 in 1,056,964,609\cite{floating_point_numbers_in_normal_interval}. Whilst this is
    astronomically unlikely, it is still technically possible and therefore should be fixed at some stage.
    \item There is a risk that if the thread inside \texttt{Worker<NodeInfo>} throws an exception, then the worker
    itself will not be in the \texttt{DEAD} state and therefore never be pruned by the \texttt{GarbageCollector} in
    \texttt{WorkerAddressBook<NodeInfo>}. This may not be the case, but it is important to investigate.
    \item Currently the \texttt{waitForAndForwardPackets} method in \texttt{WorkerDatagramForwardingScript <NodeInfo>}
    seems to assume that the last four bytes of the packet will be a source ip-address, but that might not be the
    case if the packet was sent using a simple \texttt{WorkerScript<NodeInfo>}. If this bug exists, then it risks
    blocking the fluid interoperability between Java workers and external processes, perhaps demanding some kind of
    architectural change to fully fix.
\end{itemize}
